%Plantilla para proyectos modulares de ciencias básicas.
%Para usar en los proyectos de Lic. en Física. Para otras carreras consulte a su coordinador.
%
%Edite donde se indica y sea necesario.
%
%Versión 2.0
%


\documentclass[letterpaper,12pt]{article}
\usepackage[utf8]{inputenc}
\usepackage[table]{xcolor}
\usepackage{graphicx}
\usepackage{epsfig}
\usepackage{float}
\usepackage[hidelinks]{hyperref}
\usepackage{array}
\usepackage{lastpage}
\usepackage{lipsum}
\usepackage[spanish,es-tabla]{babel} % Se considera TABLA en lugar de CUADRO en las tablas. Modificado por Rotosca
\usepackage{fancyhdr}
\usepackage{geometry}

%Matemática
\usepackage{amsmath}
\usepackage{amsfonts}
\usepackage{amssymb}
\usepackage{tabularx}
\usepackage{multicol}
\usepackage{soul}
\usepackage{amstext}


%tamaño de la fuente en las secciones.
\usepackage{sectsty}
\sectionfont{\fontsize{12}{15}\selectfont}
\subsectionfont{\fontsize{11}{15}\selectfont}


\definecolor{AzulModular}{cmyk}{1.0,0.49,0.0,0.47}%color #4365A4

\setlength{\parindent}{0.95cm}

\geometry{
  left=0.748in,
  right=0.780in,
  headsep=0.2cm,
  headheight=1.66cm,
  top=3.14cm,
%  bottom=3.54cm,
  footskip=2.54cm,
} 
\decimalpoint

\newcommand{\cfbox}[2]{%
    \colorlet{currentcolor}{.}%
    {\color{#1}%
    \fbox{\color{currentcolor}#2}}%
}


\pagestyle{fancy}
\cfoot{}
\renewcommand{\headrulewidth}{0pt}
\fancyhead[CO,LO,RO]{} %% clear out all headers
\fancyhead[C]{%
          \begin{tabular}{m{1.5cm}m{11.5cm}m{2.5cm}}
          \includegraphics[height=1.5cm]{udg_4365A4} &\cellcolor{AzulModular}
          \centering
          \cfbox{white}{%
            \begin{minipage}{10.5cm}
              \centering
              \textcolor{white}{Evaluación Modular, Departamento de Física, 2025A}
            \end{minipage}
            }
            &
          \centering
          \tiny{Licenciatura en Física\\%para usar en otras carreras consulte a su coordinador
%%%%%%%%%%%%%%%%%%%%%%%%%%%%%%%%%%%%%%%%%%%%%%%%%%%%%%%%%%%%%%%%%%%%%%%%%%%%%%%%%%%%%%%%%%%%%%%%%
            Modular: I,II,..\\ %elija el que corresponda ejemplo :  Modular : I
%%%%%%%%%%%%%%%%%%%%%%%%%%%%%%%%%%%%%%%%%%%%%%%%%%%%%%%%%%%%%%%%%%%%%%%%%%%%%%%%%%%%%%%%%%%%%%%%%
            Página \thepage \hspace{1pt} de \pageref{LastPage}\\
          }\tabularnewline
%          \hline
          \end{tabular}%
}

%%%%%%%%%%%%%%%%%%%%%%%%%%%%%%%%%%%%%%%%%%%%%%%%%%%%%%%%%%%%%%%%%%%%%%%%%%%%%%%%%%%%%%%%%%%%%%%%%
\title{\fontsize{0.5cm}{0.5cm}\selectfont
\textit{\textbf{El título del trabajo}} %\thanks{Nota al pie de página si es necesaria}
}
\author{ \fontsize{0.35cm}{0.5cm}\selectfont
\textit{ Nombre del Primer Estudiante$^1$} \\
       \and  \fontsize{0.35cm}{0.5cm}\selectfont \textit{Nombre del Segundo Estudiante$^2$} \\
        \and \fontsize{0.35cm}{0.5cm}\selectfont \textit{\textbf{Asesor 1:}Nombre del Asesor$^1$} %si solo hay un asesor borrar el ``1''
        \and \fontsize{0.35cm}{0.5cm}\selectfont \textit{\textbf{Asesor 2:}Nombre de Segundo Asesor$^2$} \\
	}
        
\date{\small
  $^1$\fontsize{0.35cm}{0.5cm}\selectfont \textit{correo@electronico\\ Departamento de Física, CUCEI, Universidad de Guadalajara\\
    Blvd. Marcelino García Barragán 1421, Col. Olímpica, Guadalajara Jal., C. P. 44430, México\\
    $^2$Otra universidad}\\[2ex]
  }
%%%%%%%%%%%%%%%%%%%%%%%%%%%%%%%%%%%%%%%%%%%%%%%%%%%%%%%%%%%%%%%%%%%%%%%%%%%%%%%%%%%%%%%%%%%%%%%%%

%%%%%%%%%%%%%%%%%%%%%%%%%%%%%%% Comienza el documento%%%%%%%%%%%%%%%%%%%%%%%%%%%%%%%%%%5
\begin{document}
\maketitle
\thispagestyle{fancy}

\section*{Resumen}
Aquí va el resumen de su trabajo... (200 palabras aprox)

\section*{Introducción}
La introducción...


\section*{Metodología}

Así se ponen ecuaciones
\begin{equation}
  e^{i\pi} + 1 = 0
  \label{eq1}
\end{equation}

Así citamos la ecuación de arriba ec.~\ref{eq1} (véase el código)

\section*{Resultados}

Así se pone una imagen
\begin{figure}[H]
   \centering
   \includegraphics[width=0.35\textwidth]{euler}
   %lo que está entre corchetes dice que el tamaño de la imagen es 35%
   %el tamaño del ancho total del ancho del texto este puede ser
   %modificada a gusto y tamaño de la imagen
   \caption{Pie de imagen}
   \label{fig:imagen1} %Para poder referenciar
\end{figure}

Así citamos una figura  Fig.~\ref{fig:imagen1} (véase el código)


% Ejemplo de tabla. Modificado por Rotosca 
\begin{table}[H] % Quiero mi gráfica aquí
    \begin{center}
	    \begin{tabular}{|l|l|}
	    	\hline
	    	\multicolumn{2}{|c|}{\text{Title}} \\
	        \hline \hline 
	        a & b \\ 
	        \hline 
	        1 & 2 \\ 
	        \hline 
        \end{tabular} 
    \end{center}
	\label{tab1}
	\caption{la tabla}
\end{table}


\section*{Conclusiones}

Para citar en el texto algo de alguna bibliografía, le ponemos
\cite{refer1} (véase el código)
     
\begin{thebibliography}{99}
    
% Se considera \emph{text} en lugar de \textit{text}. Modificado por Rotosca
\bibitem{refer1} \fontsize{0.35cm}{0.5cm}\selectfont{Nombre de los
  Autores, `` Título del artículo o sección del libro consultado''
  \emph{Nombre de la revista o libro}, Vol. X, No. X, p. XX-XX, Año
  de la publicación.}

  \end{thebibliography}

\vspace{2cm}

\rule{75mm}{0.1mm}\hfill\rule{75mm}{0.1mm}


Nombre y Firma del Asesor$^1$ \hspace{3.5cm} Nombre y Firma de Segundo Asesor$^2$
\centering 

\end{document}
